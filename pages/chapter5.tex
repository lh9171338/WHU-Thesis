% Chapter 5

\chapter{总结与展望}
\section{工作总结}
本文从空地联合在目标搜索方面的优势出发,参考国内外发展现状,利用计算机视觉技术,自主定位、建图与导航技术,自动控制技术,传感器技术等实用技术,通过硬件设计和软件开发,最终实现了一套完整的基于无人机-车的空地联合目标搜索系统。本文的主要工作与贡献如下:

\begin{enumerate}[label=(\arabic*)] 
	\item 实现了无人机系统。在硬件上,搭建了一套基于大疆M100无人机、机载电脑Manifold 、云台相机、GPS模块、蓝牙串口模块和遥控器的无人机开发平台;在软件上,实现了目标的检测与定位,以及无人机的自主飞行;
	\item 实现了无人车系统。在硬件上,搭建了一套基于Robomasters步兵车底盘、STM32控制主板、TX2机载电脑、激光雷达、姿态传感器、GPS模块、蓝牙串口模块和遥控装置的无人车开发平台;在软件上,实现了无人车的自主定位、建图、避障和导航;
	\item 实现了空地联合目标搜索。首先利用无人机扫描待搜索区域,同时进行目标检测和定位,然后利用蓝牙串口模块将目标位置信息发送给无人车,最后无人车通过自主导航行驶到目标处,完成目标搜索任务。
\end{enumerate}

实验表明,本设计中的无人机系统实现了目标检测与定位的功能,无人车系统实现了自主定位、建图和导航的功能,并且整个系统具备了空地联合目标搜索的能力。

\section{存在的问题与展望}
限于研究的时间,本文提出的基于无人机-车的空地联合搜索系统虽然已初步成型,但仍有许多不足之处,有待改进和完善。例如具体目标的检测与定位问题、依赖于GPS信号的问题、无人机与无人车的定位精度问题、复杂环境下的工作鲁棒性问题、系统在真实应用场景中的可靠性问题等等。针对存在的问题与不足,作者在此提出以下几点展望:

\begin{enumerate}[label=(\arabic*)] 
	\item 利用深度学习实现具体目标的检测,不过这需要采集大量的训练样本,同时目标定位算法也会更加复杂,可以考虑使用双目相机或深度相机进行目标定位;
	\item 利用无人机生成全局地图,用于无人机和无人车的定位,从而摆脱对GPS信号的依赖;
	\item 融合更多高性能传感器,提升无人机与无人车的定位精度,使系统更加可靠。
\end{enumerate}